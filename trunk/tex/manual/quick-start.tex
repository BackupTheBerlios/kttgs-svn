\section{Schnellstart}

OK. Ihr habt also alles installiert. Jetzt sollte man auch mal
testen, ob alles funktioniert \ldots und nat�rlich auch um seine
Neugierde zu befriedigen ;-)

F�hrt einfach die nachfolgenden Schritte aus. Falls etwas nicht 
funktionieren sollte, dann schreibt eine Mail an mich und
beschreibt die Fehlermeldung(en). Ich hoffe aber nat�rlich
inst�ndig, da� alles klappt. In dem Fall k�nnt ihr mir auch
eine Mail schreiben. Einfach um mein Ego zu st�rken \ldots

Im folgenden muss ich mich oftmals auf Verzeichnisse beziehen.
Wenn ich \texttt{./} (Punkt Slash) schreibe, dann meine ich damit 
\textbf{das} Verzeichnis, in dem ihr die Zip-Dateien ausgepackt 
habt. Alle Aktionen die nachfolgend durchgef�hrt werden, m�ssen in 
diesem Verzeichnis gestartet werden.

\subsection{Build.properties anpassen}
Wir haben eine zentrale Steuerungsdatei, die kontrolliert welche
Verzeichnisstruktur angelegt, welche Sprache und welches Layout
(f�r HTML) verwendet werden soll. Dies Datei hei�t
\texttt{./build.properties}. Bitte �ffnet und editiert diese
mit eurem Lieblingseditor\footnote{Du nimmst jetzt den \texttt{vi}? Cool!}.

Die Datei \texttt{./build.properties} sollte dann wie folgt aussehen
(die Zeilennummern hab' ich nur zur Orientierung da, sie geh�ren
\textbf{nicht} in die Datei!):

\lstset{numbers=left,numberstyle=\tiny,stepnumber=1}
\begin{lstlisting}{}
tour.name=MeineSuperTour
# Currently supported languages: de, en
lang=de
css=biketour.css
\end{lstlisting}

\subsection{Skeleton generieren}
Soweit so gut. Aber um irgendwas zu sehen brauchen wir ja
'ne Tourbeschreibung. Woher nehmen, wenn nicht zaubern? Einfach
zaubern! Das System kann ein Grundger�st (englisch: 
\textit{Skeleton}) anlegen. Dazu ist nur der nachfolgende 
Aufruf von Ant notwendig. Ab auf die Kommandozeile und
den nachfolgenden Befehl ausf�hren:

\lstset{numbers=left,numberstyle=\tiny,stepnumber=1}
\begin{lstlisting}{}
ant gen-skeleton
\end{lstlisting}

\subsection{HTML-Ausgabe erzeugen}
Das Erzeugen der HTML-Datei ist genauso einfach. Auf der
Kommandozeile tippt ihr einfach diesen Befehl:

\lstset{numbers=left,numberstyle=\tiny,stepnumber=1}
\begin{lstlisting}{}
ant gen-tour-html
\end{lstlisting}

Da k�nnten jetzt zwei Warnungen (englisch: \textit{warning(s)})
auftauchen. Die k�nnt ihr ignorieren.

\subsection{HTML im Browser bewundern}
\textit{Jetzt kommt der gro�e Moment, wo der Frosch ins Wasser 
springt.} In dem Verzeichnis \texttt{./html/MeineSuperTour} sollte 
sich eine Datei namens \texttt{MeineSuperTour.html} befinden.

�ffnet diese Datei (\texttt{./html/MeineSuperTour/MeineSuperTour.html})
in einem Browser eurer Wahl\footnote{Getestet hab' ich nur: Firefox 
und Konquerer}. Wenn ihr jetzt 'ne Beschreibung einer fiktiven Tour
seht, dann l�uft das System. \textbf{Gl�ckwunsch!}

