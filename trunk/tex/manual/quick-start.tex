\section{Schnellstart}\label{sec:qickstart}

OK. Ihr habt also alles installiert. Jetzt sollte man auch mal
testen, ob alles funktioniert \ldots und nat�rlich auch um seine
Neugierde zu befriedigen ;-)

F�hrt einfach die nachfolgenden Schritte aus. Falls etwas nicht 
funktionieren sollte, dann schreibt eine Mail an mich und
beschreibt die Fehlermeldung(en). Ich hoffe aber nat�rlich
inst�ndig, da� alles klappt. In dem Fall k�nnt ihr mir auch
eine Mail schreiben. Einfach um mein Ego zu st�rken \ldots

Im folgenden muss ich mich oftmals auf Verzeichnisse beziehen.
Wenn ich \texttt{./} (Punkt Slash) schreibe, dann meine ich damit 
\textbf{das} Verzeichnis, in dem ihr die Zip-Dateien ausgepackt 
habt. Alle Aktionen die nachfolgend durchgef�hrt werden, m�ssen in 
diesem Verzeichnis gestartet werden.

Wenn ihr in dem Verzeichnis eine Datei namens \texttt{build.xml}
seht, dann seid ihr vermutlich richtig.

Alle Aufrufe von \texttt{ant}, die hier verwendet werden, sind
im Abschnitt \ref{sec:ant-calls} ab Seite \pageref{sec:ant-calls} 
detailliert beschrieben.

\subsection{Build.properties anpassen}

Wir haben eine zentrale Steuerungsdatei, die kontrolliert welche
Verzeichnisstruktur angelegt, welche Sprache und welches Layout
(f�r HTML) verwendet werden soll.

Dies muss man konfigurieren. Man kann das nun mit einem Texteditor
tun: Das ist im Abschnitt \ref{sec:properties_with_editor} 
beschrieben.

Man kann sich das Leben aber auch leichter machen und diese
Datei mithilfe der grafischen Oberfl"ache ver"andern. Das
ist im Abschnitt \ref{sec:properties_with_gui} erl"autert.
Wenn ihr die Oberfl"ache verwendet, dann braucht ihr den
Abschnitt "uber den Editor nicht zu lesen\ldots aber
vielleicht interessiert es dich doch, wie das System funktioniert.

Die Oberfl"ache hat den Vorteil, da"s Fehler (nahezu?) ausgeschlossen
sind.

\subsubsection{Mit dem Editor}\label{sec:properties_with_editor}
Die erw"ahnte, zentrale Steuerungsdatei hei"st \texttt{./build.properties}. 
Bitte �ffnet und editiert diese mit eurem Lieblingseditor\footnote{Du 
nimmst jetzt den \texttt{vi}? Cool!}.

Die Datei \texttt{./build.properties} sollte dann wie folgt aussehen
(die Zeilennummern hab' ich nur zur Orientierung da, sie geh�ren
\textbf{nicht} in die Datei!):

\lstset{numbers=left,numberstyle=\tiny,stepnumber=1}
\begin{lstlisting}{}
tour.name=MeineSuperTour
# Currently supported languages: de, en
lang=de
css=biketour.css
\end{lstlisting}

\subsubsection{Mit der grafischen Oberfl"ache}\label{sec:properties_with_gui}

Die Oberfl"ache sollte intuitiv bedienbar sein. Du gibst
einfach die Daten ein und dr"uckst den Knopf \textsc{Create}.
Falls irgendetwas nicht korrekt ist, dann bekommst du eine
entsprechende Fehlermeldung. Wenn keine Fehlermeldung
erscheint, dann klickst du auf \textsc{OK} und \textit{der K"ase
ist gegessen}. Keine Hemmungen! Man kann nichts kaputt
machen\ldots

\textbf{Wichtig!} In dem Textfeld \textsc{Tour folder name}
gibst du bitte den Text \texttt{MeineSuperTour} ein. Wenn du da
was anderes verwendest, dann passt die nachfolgende Beschreibung
nicht. Also: Mach's dir leicht und mach' was ich dir gesagt
habe\ldots

Zum Starten der Oberfl"ache gibst du auf der Kommandozeile 
den folgenden Befehl ein:
\lstset{numbers=left,numberstyle=\tiny,stepnumber=1}
\begin{lstlisting}{}
ant run.config
\end{lstlisting}


\subsection{Skeleton generieren}\label{sec:qickstart-gen-skeleton}
Soweit so gut. Aber um irgendwas zu sehen brauchen wir ja
'ne Tourbeschreibung. Woher nehmen, wenn nicht zaubern? Einfach
zaubern! Das System kann ein Grundger�st (englisch: 
\textit{Skeleton}) anlegen. Dazu ist nur der nachfolgende 
Aufruf von Ant notwendig. Ab auf die Kommandozeile und
den nachfolgenden Befehl ausf�hren:

\lstset{numbers=left,numberstyle=\tiny,stepnumber=1}
\begin{lstlisting}{}
ant gen-skeleton
\end{lstlisting}

\subsection{HTML-Ausgabe erzeugen}
Das Erzeugen der HTML-Datei ist genauso einfach. Auf der
Kommandozeile tippt ihr einfach diesen Befehl:

\lstset{numbers=left,numberstyle=\tiny,stepnumber=1}
\begin{lstlisting}{}
ant gen-tour-html
\end{lstlisting}

Da k�nnten jetzt zwei Warnungen (englisch: \textit{warning(s)})
auftauchen. Die k�nnt ihr ignorieren.

\subsection{HTML im Browser bewundern}\label{sec:view-html}
\textit{Jetzt kommt der gro�e Moment, wo der Frosch ins Wasser 
springt.} In dem Verzeichnis \texttt{./html/MeineSuperTour} sollte 
sich eine Datei namens \texttt{MeineSuperTour.html} befinden.

�ffnet diese Datei (\texttt{./html/MeineSuperTour/MeineSuperTour.html})
in einem Browser eurer Wahl\footnote{Getestet hab' ich nur: Firefox 
und Konquerer}. Wenn ihr jetzt 'ne Beschreibung einer fiktiven Tour
seht, dann l�uft das System. \textbf{Gl"uckwunsch!}

\subsection{Die Tour per eMail verschicken}
So, die Tour\footnote{Wie man die Tour erfasst wird im Abschnitt 
\ref{sec:xml-structure} ab Seite \pageref{sec:xml-structure} beschrieben.} 
h"atten wir nun. Aber wie kann man die Beschreibung einem Freund 
zukommen lassen? Auch das ist einfach, man packt alles in eine gezippte
Datei und verschickt sie per eMail. Wie gehen mal wieder auf die
Kommandozeile und geben den nachfolgenden Befehl ein:

\lstset{numbers=left,numberstyle=\tiny,stepnumber=1}
\begin{lstlisting}{}
ant zip-tour-html
\end{lstlisting}

Jetzt sollte im aktuellen Verzeichnis eine gezippte Datei
entstanden sein. In unserem (Beispiel-)Fall hei"st sie
\texttt{MeineSuperTour.zip}. Sie beinhaltet alles, was man
zur Betrachtung der Tour im HTML-Browser ben"otigt. Der
Empf"anger der eMail mu"s nur die Datei in einem neu
angelegten Verzeichnis auspacken und kann sich die 
Tourbeschreibung angucken. 