\section{Der Aufbau der  XML-Datei}\label{sec:xml-structure} 

Der gesamte Inhalt einer Tourbeschreibung wird in einer XML-Datei abgelegt.
Ich werde in diesem Kapitel detailliert beschreiben, wie diese Datei aufgebaut
ist.

Das Wissen um den Aufbau dieser Datei ist aus zwei Gr"unden notwendig:
\begin{enumerate}
\item Solange wir keine grafische Schnittstelle zur Erfassung besitzen, mu"s
      man diese Datei per Hand erfassen. Dazu mu"s man nat"urlich wissen, wie
      sie aufgebaut ist.
\item Auch wenn wir irgenwann eine grafische Schnittstelle haben, mu"s ein(e)
      Entwickler(in), die/der das System erweitern will, wissen, wie der Aufbau
      nun exakt aussieht\footnote{Ok, es existiert eine dokumentierte DTD, aber 
      das nur am Rande\ldots}.
\end{enumerate}

Sieht man sich die XML-Datei das erste Mal an, dann schaut das alles recht 
komplex und konfus aus. Aber, ehrlich, so schwierig ist das alles gar nicht.
Fangen wir mal an\ldots

\textit{Anm.: In den XML-Beispielen sind Zeilennummern erg"anzt. Diese geh"oren 
nat"urlich \textbf{nicht} in die echte Datei!
}

\subsection{Learning By Doing}
Das, was ich nun beschreiben werde, ist etwas trocken. Ich hab' zwar versucht,
es etwas aufzulockern, aber es mu"s ja trotzdem technisch korrekt sein und
das f"uhrt zwangsl"aufig zu einer gewissen \textit{Trockenheit}.

Interessanter is es sicher, die ganzen Sachen w"ahrend des Lesens auch
auszuprobieren. Dann ist das Ganze nicht ganz so trocken und man sieht
gleich, ob man alles richtig verstanden hat.

Das ist sehr einfach mit dem System m"oglich: Als erstes legt man mal ein
Skeleton an, wie ich es in Abschnitt \ref{sec:qickstart-gen-skeleton} 
(Seite \pageref{sec:qickstart-gen-skeleton}) beschrieben habe. Falls
du das noch nicht getan hast, dann solltest du jetzt wirklich mal
den kompletten Abschnitt \ref{sec:qickstart} lesen.

Nun kannst du mit dem Skeleton experimentieren. Hast du die Namen aus 
Abschnitt \ref{sec:qickstart} verwendet, dann befindet sich eine 
Datei names \texttt{MeineSuperTour.xml} im Verzeichnis 
\texttt{./xml/MeineSuperTour}. Dies ist die XML-Datei, um die sich
alles dreht. Du editierst diese Datei und "uberpr"ufst die "Anderungen, 
indem du folgendes ausf"uhrst:
\lstset{numbers=left,numberstyle=\tiny,stepnumber=1}
\begin{lstlisting}{}
ant gen-tour-html
\end{lstlisting}

Um die "Anderungen im Browser sichtbar zu machen, musst du die
Ansicht in deinem Browser aktualisieren. Welche Datei du 
in den Browser laden musst, ist in Abschnitt \ref{sec:view-html}
auf Seite \pageref{sec:view-html} beschrieben.

\subsection{<tourguide>}

Die gesamte Tourbeschreibung wird in zwischen dem "offnenden und dem
schlie"senden \texttt{<tourguide>}-Tag eingeschlo"sen. Das Ganze sieht also
so aus:

\lstset{numbers=left,numberstyle=\tiny,stepnumber=1}
\begin{lstlisting}{}
<tourguide>
   <storage-info>
      <location />
   </storage-info>
   <general>
      ...
   <general>
   <crosspoints>
      ...
   </crosspoints>
</tourguide>
\end{lstlisting}

Im Wesentlich haben wir also drei Abschnitte:
\begin{enumerate}
\item <storage-info>
\item <general>
\item <crosspoints>
\end{enumerate}

Alle Abschnitte werden nachfolgenden genauer beleuchtet. Der
Wichtigste is \texttt{<crosspoints>}, da er die eigentlich
Beschreibung der Tour beinhaltet. Wozu sind aber die anderen
da? Fangen wir an mit\ldots

\subsection{<storage-info> (<location>)}
Wie schon erw"ahnt, erfassen wir ja unsere Tour nicht f"ur uns 
selbst: Wir kennen sie ja schon. Sondern f"ur andere.

Mal angenommen, du wohnst in Hamburg und f"ahrst "ubers Wochende
nach Buttenheim in Bayern (das MTB ist nat"urlich dabei). Am
Samstag h"attest du gute drei Stunde Zeit 'ne kleine Tour zu
fahren. Gibt's da bekannte Touren?

W"are jetzt nett, einfach ins Internet zu gehen. Das dir 
bekannte Interface zu benutzen und mal zu gucken, was da in
Buttenheim so geht.

Genau daf"ur ist das Tag \texttt{<storage-info><location></storage-info>}
vorhanden. Wie? Schauen wir uns das mal an:

\lstset{numbers=left,numberstyle=\tiny,stepnumber=1}
\begin{lstlisting}{}
<tourguide>
   <storage-info>
      <location 
          country="de" 
          region-or-state="Bavaria" 
          city="Buttenheim" 
          zip-code="96155"/>
   </storage-info>
   ...
</tourguide>
\end{lstlisting}

Alles klar? Alle Infos werden mithilfe der Attribute des Tags
\texttt{<location>} erfa"st. Wie werden diese Attribute nun 
zur Suche benutzt? Sorry, das ist (noch?) nicht Aufgabe dieses
Projekts. Sie sind erstmal da. Es liegt nun an demjenigen, der
die Suchfunktion implementiert, wie clever die erfassten Daten
genutzt werden.

Eine kurze Anmerkung zum Attribut \texttt{country}: Es sollten
die sogenannten Top Level Domains (wie sie aus dem Internet
bekannt sind) verwendet werden. Ein einfacheres und bekannteres 
System zur Angabe von L"andern existiert praktisch nicht.

Gut. Nun haben wir (mit viel Gl"uck) drei Touren gefunden.
Aber sind wir daf"ur auch fit genug? Dauert das nicht zu 
lange? Wie kommt man an den Startpunkt? Gibt's ne Karte?
Ja, das braucht man alles. Deswegen gibt es das Tag
\texttt{<general>}\ldots

\subsection{<storage-info> (<authoring>)}
Nat"urlich sollte sich auch der Autor der Tour verewigen
k"onnen. Dazu dient das Tag <authoring> innerhalb von
<storage-info>.

Im Attribute \texttt{author} sollte der Autor seine eMail-Adresse
angeben. Dann kann man ihm bei Unklarheiten eine eMail
schicken.

Man sollte auch im Attribute \texttt{creation-date} das
Erstellungsdatum der Tour angeben. Es k"onnte ja sein, da"s
jemand die Tour 10 Jahre nach der Erfassung abfahren will.
Dann k"onnte es durchaus sein, da"s nicht mehr alle Angaben
(auch eine 100-j"ahrige Eiche kann mal gef"allt worden
sein) so stimmen. Der Leser kann sich dann darauf 
einstellen.

Als Datumsformat wird das Muster \texttt{YYYYMMDD}
\footnote{4stelliges Jahr, 2stelliges Monat und 2stelliger Tag.}
verwendet. D.h., da"s der 27. Januar 2006 als 
\texttt{creation-date="20062701"} notiert wird.


\lstset{numbers=left,numberstyle=\tiny,stepnumber=1}
\begin{lstlisting}{}
<storage-info>
   <location ... />
          
   <authoring 
      author="email@not-real.com"
      creation-date="YYYYMMDD"   />
</storage-info>
\end{lstlisting}{}

\subsection{Ein Wort zu Bildern}\label{sec:image_references}
Hmm, den Abschnitt \texttt{<general>} hab' ich ins n"achste
Kapitel verschoben. Ich mu"s noch schnell was zu Bildern
im Allgemeinen sagen. Dauert nicht lange, ist aber wichtig
f"ur \textbf{alle} nachfolgenden Abschnitte.

Nat"urlich unterst"utzt das System die Verwendung von Bildern.
Ohne Bilder w"are das Ganze wohl nur, wenn "uberhaupt, die
H"alfte wert. Dazu m"ussen die Bilder aber auch  gefunden
werden. Aber wie?

So: In dem Verzeichnis in dem die XML-Datei, "uber die wir
hier ja reden, zu liegen kommt, existiert \footnote{Wenn das
das Grundger"ust via \texttt{ant gen-skeleton} angelegt wurde,
dann existiert es automatisch. Ansonsten mu"s es per Hand
z.B. mit dem Befehl \texttt{mkdir images} unter linux-artigen
Systemen bzw. \texttt{md images} auf einem Windows-Rechner
angelegt werden.} ein Unterverzeichnis namens \texttt{images}.
Hier hinein werden alle Bilder kopiert, die wir sp"ater
ben"otigen.

Gut, jetzt h"atten wird alle Bilder in unserem Verzeichnis.
Aber wie geben wir sie in der XML-Datei an? Auch das ist
ziemlich einfach. Schauen wir uns mal ein Beispiel an:
\lstset{numbers=left,numberstyle=\tiny,stepnumber=1}

\lstset{numbers=left,numberstyle=\tiny,stepnumber=1}
\begin{lstlisting}{}
<image name="big_drop.jpg"/>
\end{lstlisting}

Das war's auch schon. Die Bilder werden \textbf{ohne} Angabe
des Verzeichnisses (\texttt{images}) durch die Notation des 
Bildnamens eingetragen. An welchen Stellen das n"otig/m"oeglich
ist, werden wir im weiteren Verlauf erfahren.

\subsection{<general>}
Ok. Nach dem Exkurs zur Angabe von Bildern, widmen wir uns
jetzt dem n"achsten Abschnitt der XML-Datei: \texttt{<general>}.
Erstmal wieder ein Beispiel:

\lstset{numbers=left,numberstyle=\tiny,stepnumber=1}
\begin{lstlisting}{}
<tourguide>
   ...
<general 
   fitness-level="easy" 
   tech-level="medium" 
   type="round" 
   duration="01:00hrs"
   total-uphill="600m">
   
   <name>Nuernberger Tiergarten 
         --- Kleiner Trial inklusive 'Roller Coaster'
   </name>
   <distance unit="km">10</distance>
   <desc>
       Wir starten am Tiergarten, fahren etwas uphill durch 
       die Stromschneisse, dann den 'Mini-DH' runter...
   </desc>
   <reach>
      <option name="Mit dem Auto">Auf der A3 aus Richtung 
         Wuerzburg oder Passau kommend bei Nuernberg/Moegeldorf 
         abfahren...
      </option>
      <option name="Per Zug">Am Nuernberg Hbf aussteigen. Dann 
         mit der Strassenbahn direkt zum Tiergarten.
      </option>
   </reach>
   <maps>
      <map>Kompass, Karte 170</map>
   </maps>
   <roadmap image="map.jpg"/>
   <profile />
</general>   
   ...
</tourguide>
\end{lstlisting}

\subsubsection{Die Attribute von <general>}
Wie man aus dem Beispiel ersieht, hat <general> vier
Attribute. Diese dienen dem/r Anwender(in) dazu, zu 
entscheiden, ob die Tour f"ur ihn/sie geeignet ist.

Die Angaben hier sind nicht ganz einfach, aber wichtig.
Wenn du topfit bist, die Tour aber auch f"ur dich
hart war, dann bringt das hier zum Ausdruck. Wenn
der Downhill f"ur dich nicht ganz einfach, aber du
technisch nicht so perfekt bist, dann setze den 
\texttt{tech-level} nicht gleich auf \texttt{extrem}.

Nachfolgend die Attribute und ihre Bedeutung (was man
hierf"ur angeben kann, wird im Anschlu"s erl"autert):
\begin{enumerate}
\item fitness-level: Wie fit muss man f"uer die Tour sein?
\item tech-level: Wie hoch ist der technische Anspruch?
      Knifflige Wurzelpassagen? Extreme Downhills (mit
      fetten Drops)?
\item type: Rundtour, Single Trial, \ldots
\item duration: Dauer der Tour. Als du sie gefahren
      bist.
\item total-uphill: Hoehenmeter der Tour.
\end{enumerate}

Werte f"ur \textbf{fitness-level}: \texttt{easy, medium, high} 
oder \texttt{extrem}.

Werte f"ur \textbf{tech-level}: \texttt{easy, medium, high} 
oder \texttt{extrem}.

Werte f"ur \textbf{type}:
\begin{itemize}
\item round $\Rightarrow$ Rundtour
\item pointtopoint $\Rightarrow$ Tour von A nach B
\item downhill $\Rightarrow$ Reiner Downhill
\item trail $\Rightarrow$ Sch"oner Singletrail
\end{itemize}

\textbf{duration}: Hier sollte man sowas wie \texttt{"01:30hrs"}
oder \texttt{"1 Stunde, 30 Minuten"} angeben.

\textbf{total-uphill}: Hiermit werden die gesamten H"ohenmeter
der Tour angegeben. Also sowas wie \texttt{"600m"}. Vergesst die
die Einheit wie z.B. \texttt{m} f"ur Meter nicht. Es koennte
sich ja auch um englische Fu"s handeln.

So und nun zu den restlichen Tags die innerhalb von
<general> verwendet werden k"onnen oder m"ussen\ldots

\subsubsection{<name>}
Hier vergibst du einen kurzen, pr"agnanten Namen f"ur die
Tour. Dieser Text taucht zum Beispiel in Listen auf, wenn man
aus mehreren Touren eine ausw"ahlen mu"s.

\subsubsection{<distance>}
Dieses Tag dient zu Angabe der Gesamtdistanz der Tour, die 
dein Fahrradcomputer angezeigt hat, als du sie gefahren bist.
Es wird folgenderma"sen verwendet:
\lstset{numbers=left,numberstyle=\tiny,stepnumber=1}
\begin{lstlisting}{}
<tourguide>
   ...
<general 
   ...
   <distance unit="km">10</distance>
   ...
</general>   
   ...
</tourguide>
\end{lstlisting}
Mit dem Attribute \texttt{unit} gibst du die L"angeneinheit
an (also z.B.: km, meter, Meilen). Zwischen den beiden
Tags notiert man dann die L"ange der Tour als Zahl.
Und das war's auch schon.


\subsubsection{<desc>}
Zwischen den <desc>-Tags wird eine Beschreibung der Tour 
erfasst. Die kann durchaus etwas l"anger sein und soll dem
Leser richtig Lust auf die Tour machen.

Die einzelnen Abs"atze werden durch <para> eingeschlossen.
Damit kann man (im Moment nur) die Ausgabe der HTML-Datei
etwas h"ubscher formatieren. Das Tag muss aber benutzt werden.
Auch wenn nur \texttt{ein} Absatz vorliegt! Hat man das Tag
vergessen, dann erscheint in der Ausgabe keinerlei Text.

\subsubsection{<reach>}
Irgendwie mu"s der Biker ja auch an den Startpunkt der Tour
gelangen. Dazu verwenden wir das <reach>-Tag. Man kann
mehrere Verkehrsmittel angeben. Ein Beispiel k"oennte wie
folgt aussehen:
\lstset{numbers=left,numberstyle=\tiny,stepnumber=1}
\begin{lstlisting}{}
<tourguide>
   ...
<general 
   ...
   <reach>
      <option name="Mit dem Auto">Auf der A3 aus Richtung 
         Wuerzburg oder Passau kommend bei Nuernberg/Moegeldorf 
         abfahren...
      </option>
      <option name="Per Zug">Am Nuernberg Hbf aussteigen. Dann 
         mit der Strassenbahn direkt zum Tiergarten.
      </option>
   </reach>
   ...
</general>   
   ...
</tourguide>
\end{lstlisting}

Innerhalb des <reach>-Tags kann man beliebig oft das Tag
<option> verwenden. Das <option>-Tag dient dazu, die 
verschiedenen Verkehrsmittel bzw. Anreisewege zu beschreiben.
Das Attribute \texttt{name} verwendet man dazu, den Namen
des Verkehrsmittel zu notieren. Innerhalb des <option>-Tag
erfolgt dann die Erkl"arung zur Anreise.

\subsubsection{<maps>}
Zum Abradeln der Tour wird man sicher 'ne Landkarte des
Gebiets gebrauchen k"onnen. Auch diese Information sollten
wir dem Leser mitgeben. Innerhalt des <maps>-Tags kannst du
mit dem <map>-Tag beliebig viele Karten angeben:
\lstset{numbers=left,numberstyle=\tiny,stepnumber=1}
\begin{lstlisting}{}
<tourguide>
   ...
<general>
   ....
   <maps>
      <map>Kompass, Karte 170</map>
      <map>Andere Firma, Andere Karte 21</map>
      ...
   </maps>
   ...
</general>   
   ...
</tourguide>
\end{lstlisting}

\subsubsection{<roadmaps>}
Du hast auch eine digitalisierte Karte der Tour (mit eingezeichnetem
Weg der Tour)? Na, dann sollten wir die auch anzeigen. Dazu
gen"ugt es, den Namen des Bildes im \texttt{image} Attribute
des <roadmap>-Tags anzugeben:
\lstset{numbers=left,numberstyle=\tiny,stepnumber=1}
\begin{lstlisting}{}
<tourguide>
   ...
<general>
   ...
   <roadmap image="map.jpg"/>
   ...
</general>   
   ...
</tourguide>
\end{lstlisting}

Aber Achtung! Kartenmaterial ist urheberlich gesch"utzt. Du
solltest (ohne Genehmigung) nicht einfach eine Karte einscannen
und deiner Tour beilegen. Sofern du die Karte (wie auch immer)
selbst gezeichnet oder eine Genehmigung hast, dann kannst du
sie ohne Bedenken verwenden.

\subsubsection{<profile/>}
Das ist ein seltsames Tag: Man muss es n"amlich einfach nur
hinschreiben\ldots und man kann es auch einfach weglassen. Aber
dann fehlt eine wichtige Information, die noch dazu automatisch
generiert werden wuerde: Das grafische Profil der Tour.

Da das Tag allein f"ur sich steht sollte man den Slash/Schr"agstrich
(\texttt{.../>}) vor der schlie"senden Klammer nicht vergessen!

\subsection{<crosspoints>}
Mann-o-mann, so 'ne Menge Zeugs zu tippen. Wann kommt jetzt
eigentlich die Beschreibung der Tour? Voila\ldots jetzt!

Innerhalb des <crosspoints>-Abschnitts werden alle relevanten
Wegpunkte der Tour beschrieben (und, optional, noch vieles
mehr).

Hier mal ein Beispiel:
\lstset{numbers=left,numberstyle=\tiny,stepnumber=1}
\begin{lstlisting}{}
<tourguide>
...
<crosspoints>
   <crosspoint 
      distance="300" elevation="600" 
      direction="northwest" 
      latitude="000" longitude="000">
     <desc><para>Ein nordwest Punkt.</para></desc>
     <profile-desc>
        Die Stromschneisse... und los
     </profile-desc>
   </crosspoint>
   
   <track-info pavement="road">
      <anecdote distance="gesamter Weg" image="drop.jpg">
          Wenn's gar nicht mehr geht, dann einfach ab in 
          den Wald. Der ist nicht dicht... man kann da 
          auch gut schieben (und wird nicht gesehen ;-).
      </anecdote>
   </track-info>
   
   <crosspoint distance="200" elevation="310" 
               direction="none" 
               latitude="000" longitude="000">
     <desc><para>Back At Home!</para></desc>
   </crosspoint>
</crosspoints>
</tourguide>
\end{lstlisting}

WTF\footnote{Engl.: \textit{What the fuck}, Ausruf des Erstaunten\ldots ;-)}?
Jaja, schaut kompliziert aus, is' aber auch blo"s mit Wasser
gekocht.

Gehen wir das mal analytisch an. Innerhalb von <crosspoints> 
scheint es ein paar andere relevante Tags zu geben. Na, also:
Richtig erkannt. Und es handelt sich dabei nur um zwei
\begin{itemize}
\item <crosspoint> und
\item <track-info>
\end{itemize}


\begin{footnotesize}
\textit{Anm. d. A.: Dem h"aufigen Leser technischer Dokumente wird nun auffallen --- 
und ihn vielleicht verwirren ---, da"s ich die Einschachtelungstiefe des XML-Dokuments
auch durch die Struktur dieses Dokuments wiedergegeben habe. Dies wird nun bewu"st 
durchbrochen. Es ist der schiere Umfung und die Strukturtiefe der nachfolgenden 
zwei Abschnitte, die dieses Vorgehen vern"unftig erscheinen lassen.}
\end{footnotesize}


\subsection{Entfernungen}\label{sec:crosspoints-distances-explained}
Folgt man 'ner fremden Tour, dann ist es nat"urlich wichtig zu
wissen, wo auf der Strecke man gerade ist. Viele Touren geben dazu
die Positionen der Kreuzungspunkte in der totalen Distanz zum
Startpunkt an. Das funktioniert nat"urlich theoretisch auch: In
einer perfekten Welt.

Aber in einer perfekten Welt misst mein Fahrradcomputer auch perfekt.
Und: Ein perfekter Fahrer verf"ahrt sich nie.

Meine Realit"at sieht aber so aus: Bis zum Kreuzungspunkt 3 hat alles
gepasst. Den Kreuzungspunkt 4 finde ich aber nicht. Ich bin schon 
1000 Meter darueber. Also: Zur"uck zum Punkt 3. Ok, ich nehm' den
anderen Weg. Nun bin ich also 2x1000 Meter "uber der Angabe in der
Tourbeschreibung. Das muss ich nat"urlich bei allen weiteren Angaben
in Betracht ziehen.

knottyTom's Tour Guide System zaehlt deswegen anders. Alle Entfernungen
werden im Abstand zum letzten Kreuzungspunkt angegeben.

\subsection{<crosspoint>}
Das <crosspoint>-Tag dient dazu, einen Kreuzungspunkt im
Verlauf der Tour zu beschreiben. Dazu verwenden wir f"unf
Attribute und zwei Subtags.

\lstset{numbers=left,numberstyle=\tiny,stepnumber=1}
\begin{lstlisting}{}
<tourguide>
...
<crosspoints>
   <crosspoint 
      distance="300" elevation="600" 
      direction="northwest" 
      latitude="000" longitude="000"
      showInProfile="no">
      
     <desc><para>Ein nordwest Punkt.</para></desc>
     <profile-desc>
        Die Stromschneisse... und los.
     </profile-desc>
   </crosspoint>
   
   <track-info pavement="road">
      ....
   </track-info>
   
   <crosspoint distance="200" elevation="310" 
               direction="none" 
               latitude="000" longitude="000">
     <desc><para>Back At Home!</para></desc>
   </crosspoint>
</crosspoints>
</tourguide>
\end{lstlisting}

\subsubsection{<crosspoint> Attribute}\label{sec:crosspoint-attributes}

Die Attribute (und ihre Bedeutung) lauten wie folgt.
\begin{itemize}
\item \texttt{distance}: Das ist der Abstand zum letzten Kreuzungspunkt
      in \textbf{Metern}. Der erste <crosspoint> hat also immer die
      \texttt{distance="0"} (Null)!
\item \texttt{elevation}: H"ohe des Punktes in \textbf{Metern}.
\item \texttt{direction}: Hiermit gibt man die Richtung an, in die es nach
      diesem Punkt weitergeht, mit einer von acht Himmelsrichtungen
      an. M"ogliche Werte sind: \texttt{north, east, south, west, northeast,
      southeast, southwest} und \texttt{northwest}.
\item \texttt{latitude}: Breitengrad (\textit{wird bisher nicht verwendet})
\item \texttt{longitude}: L"angengrad (\textit{wird bisher nicht verwendet})
\item \texttt{showInProfile}: Dieses Attribute gibt an, ob der 
      Kreuzungspunkt im grafischen Profil als Nummer erscheinen soll. Wird
      das Attribute nicht verwendet, dann erscheint der Punkt im Profile.
      Das ist gleichedeutend mit \texttt{showInProfile="yes"}. Liegen die
      die Kreuzungspunkt im grafischen Profil sehr dicht beisammen, dann
      sieht das nicht gut aus und ist auch nicht gut zu lesen. In diesem
      Fall sollte man \texttt{showInProfile="no"} angeben, dann wir die
      Nummer weggelassen. Einfach mal ausprobieren! ;-)
\end{itemize}

\subsubsection{<desc>}
Dieses Tag dient der Beschreibung des Kreu\-zungs\-punktes. Das ist,
auch wenn man das nicht gleich erkennt, eines der wichtigsten Tags
der ganzen Beschreibung! 

An so einem Punkt mu"s sicher Fahrer wieder neu orientieren und
"uberpr"ufen, ob er noch auf der richtigen Route ist. Gib ihm/ihr
an dieser Stelle Hinweise, wie es jetzt weitergeht. Vorallem dann,
wenn sich der Weg gabelt oder es irgendwo nicht leicht ersichtlich 
in den Wald geht. Du solltest also so Angaben machen wie:
\begin{itemize}
\item \textit{Hier gabelt sich der Weg in drei Schotterwege. Wir
   nehmen den mittleren.}
\item \textit{Hinter der alten Eiche geht es rechts in den Waldweg.}      
\item \textit{Bisher waren wir auf "gelb Kreis", jetzt fahren wir auf
   "rot Viereck" weiter.}
\end{itemize}

Jeden Absatz muss man wieder innerhalb von <para>\dots</para>
schreiben. Vergisst man die <para>-Tags, dann sieht man wieder
keine Textausgabe im generierten Dokument!

\subsubsection{<images>}
Innerhalb des <images>-Tag kann man bieliebig viel Bilder, die
f"ur den Kreuzungspunkt relevant sind, angeben. 

\lstset{numbers=left,numberstyle=\tiny,stepnumber=1}
\begin{lstlisting}{}
<tourguide>
...
<crosspoints>
   <crosspoint...> 
     <desc><para>Ein nordwest Punkt.</para></desc>
     <images>
        <image name="einBild.jpg"/>
        <image name="anderesBild.jpg"/>
        ....
     <images>
   </crosspoint>
   ....
</crosspoints>
</tourguide>
\end{lstlisting}

\subsubsection{<profile-desc>}
Wie ich ja schon erw"ahnt habe, kann das System automatisch 
ein grafisches Profil der Tour aus den erfassten Daten generieren.
Dazu geh"ort auch eine Legende, die die einzelnen Punkt des Profil
kommentiert.

Dazu wird normalerweise der Text aus dem <desc>-Tag verwendet. Dieser
kann aber oftmals zu lang oder einfach unpassend f"ur eine Legende
sein. 

Deshalb kann man mithilfe des <profile-desc>-Tags diesen Text
"uberschreiben. Das Tag ist zwar optional, ich w"urde aber dennoch
Gebrauch davon machen.

\subsection{<track-info>}
Innerhalb des <trackinfo>-Tags werden Daten erfa"st, die zwischen
zwei Kreu\-zungs\-punkten von Interesse f"uer den Fahrer der Tour sind,
bzw. sein k"onnten.

Dazu geh"oren z.B. Sachen wie: Wie schaut der Stra"senbelag aus?
Wo gibt's was zu futtern? Gibt's auch was zu sehen? Geschichten zur 
eigenen Tour.

Den Stra"senbelag gibt man im Attribute \texttt{pavement} des
Tags an also z.B. so \texttt{<trackinfo pavement="trail">} an.
Erlaubte Werte und ihre Bedeutung sind:
\begin{itemize}
\item \texttt{road} $\Rightarrow$ Asphaltierte Stra"se
\item \texttt{forrest.road} $\Rightarrow$ Wald- oder Feldweg
\item \texttt{trail} $\Rightarrow$ Schmaler Wanderweg
\end{itemize}

Das Ganze sieht also z.B. wie folgt aus:
\begin{lstlisting}{}
<tourguide>
...
<crosspoints>
   ...
   <track-info pavement="road">
      ...
   </track-info>
   ...
</crosspoints>
</tourguide>
\end{lstlisting}

Innerhalb des <track-info>-Tags sind die Tags <eatndrink>, <poi> 
und <anecdote> erlaubt, aber allesamt sind optional. Sie haben alle 
zwei einheitliche Attribute. Sehen wir uns die als Erstes an.

\subsubsection{Attribute f"ur <eatndrink>,<poi> und <anecdote>}
Wir haben die beiden Attribute:
\begin{itemize}
\item \texttt{distance} und
\item \texttt{image} (optional)
\end{itemize}

Mit \texttt{distance} wird die Entfernung zum letzten Kreuzungspunkt
angegeben. Das ist jetzt \textbf{nicht} der Kreuzungspunkt zu dem
dem dieser \texttt{<track-info>}-Abschnitt geh"ort, sondern der
\textbf{vor} diesem!
Genaueres dazu findest du in Absschnitt 
\ref{sec:crosspoints-distances-explained} auf Seite 
\pageref{sec:crosspoints-distances-explained}.

\texttt{image} dient dazu ein Bild anzugeben. Es gilt alles
das, was ich in Abschnitt \ref{sec:image_references} auf Seite
\pageref{sec:image_references} zu Bildern geschrieben
habe.


\subsubsection{<eatndrink>}
Die Leute die deine Tour abradeln, werden sicher Hunger und
Durst bekommen. Hilf ihnen und beschreib', wo es was zu
Futtern gibt. Die Beschreibung erfolgt innerhalb der Tags. Hier 
ein Beispiel:
\begin{lstlisting}{}
<eatndrink distance="1200" image="dead_sow.jpg">
    "Zur Toten Sau" ist ein uriges und guenstiges fraenkisches 
    Lokal. Man sollte unbedingt das "Schaeufala" probieren.
    Fuer Vegetarier... ne, fuer solche Menschen gibt es da
    aber schon gar nix. >;-)
</eatndrink>
\end{lstlisting}


\subsubsection{<poi>}
POI steht fuer \textit{point of interest}, also ein 'interessanter
Punkt'. Das kann nun sehr vieles sein: Ein uralte Eiche. Reste eines
r"omischen Walls. Ein Wasserfall. Eine Felsformation.

Ich finde man sollte mit dem MTB nicht nur durch die Gegend rasen,
sondern sich auch (hin und wieder) die Gegend ansehen, durch die man
f"ahrt. Wenn ich nur m"oglichst schnell unterwegs sein m"ochte  dann 
k"onnte ich mir ja auch ein Rennrad anschaffen\ldots 

Mit <poi> kann man auf solche Punkte hinweisen, z.B. so:
\begin{lstlisting}{}
<poi distance="400" image="rocks.jpg">
    Hier befindet man sich auf der sog. "Retterner Kanzel".
    Man sollte sich wirklich die Zeit nehmen und mal fuer
    5 Minuten "ins Land" zu blicken...
</poi>
\end{lstlisting}

\subsubsection{<anecdote>}
Ihr habt sicher ein paar Sachen beim Erkunden und Erfassen
der Tour erlebt, die der Nachwelt erhalten bleiben sollten,
auch wenn sie nicht relevant f"ur die Tour selber sind.
Egal, notiert auch das im <anecdote>-Tag z.B. wie folgt:
\begin{lstlisting}{}
<anecdote distance="640" image="broken_bike.jpg">
    Bei der Abfahrt hat es mich so zerlegt, dass ich mein
    Bike wegschmeissen konnte. Und Martin hatte nichts
    Besseres zu tun als mein Leid auch noch zu 
    fotografieren... ;-)
</anecdote>
\end{lstlisting}

\subsubsection{<profile-points>}
Wenn man eine Tour erfasst, dann m"ussen die H"ohenangaben
an den Kreuzungs\-punkten nicht unbedingt das echte Profil
widerspiegeln. Es kann durchaus vorkommen, da"s zwischen
zwei Kreuzungspunkten ein echter Hammer liegt.

Das k"onnen wir bisher aber nicht erfassen. Deswegen gibt
es am Ende des \texttt{<track-info>}-Abschnitts den 
\texttt{<profile-points>}-Bereich. Hier k"onnen 
Profilpunkte notiert werden, die zwischen zwei Kreuzungs\-punkten
auftreten.

Das sieht dann z.B. so aus:
\begin{lstlisting}{}
<tourguide>
...
<crosspoints>
   ...
   <track-info pavement="road">
      ...
      <profile-points>
        <point distance="500" elevation="300"
	       latitude="000" longitude="000"
	       icp="false"/>
	<point distance="1000" elevation="150"
	       latitude="000" longitude="000"
	       icp="false"/>
      </profile-points>
   </track-info>
   ...
</crosspoints>
</tourguide>
\end{lstlisting}

Jeder Profilpunkt wird mit einem \texttt{<point>}-Tag erfasst.
Die Bedeutung der Attribute \texttt{distance, elevation, latitude}
und \texttt{longitude} entspricht der eines \texttt{<crosspoint>}-Tags
(siehe dazu auf den Punkt \ref{sec:crosspoint-attributes} auf Seite
\pageref{sec:crosspoint-attributes}). Neu ist eigentlich nur das
Attribute \texttt{icp}. Es muss immer vorhanden sein und immer
den Wert \texttt{false} haben. Das ist seltsam. Stimmt! Das ist
im Moment aber aus technischen Gr"unden n"otig.

Mit \texttt{distance} wird die Entfernung zum letzten Kreuzungspunkt
angegeben. Das ist jetzt \textbf{nicht} der Kreuzungspunkt zu dem
dem dieser \texttt{<track-info>}-Abschnitt geh"ort, sondern der
\textbf{vor} diesem! Genaueres dazu findest du in Absschnitt 
\ref{sec:crosspoints-distances-explained} auf Seite 
\pageref{sec:crosspoints-distances-explained}.

%%%%%%%%%%%%%%%%
%% The END
\subsection{Keine Tags mehr!}
Das war heftig, oder? Aber ich hab' eine positive "Uberraschung:
Das waren alle Tags. Jetzt solltest du alles kennen was zur 
Erfassung einer Tour notwendig ist.

Falls du dennoch Probleme haben solltest, dann schick' mir
halt mal eine eMail an die Adresse, die auf dem Deckblatt
angegeben ist. Vielleicht kann ich dir ne L"osung schicken
und diese Dokumentation verbessern.

Falls du es geschafft hast, eine Tour zu erfassen, dann kannst
du die zip-Datei der Tour vielleicht auch irgendwo im Internet
hochladen. Ich versuche einen Index aller Touren aufzubauen,
die mit \textsc{kttgs} erfassst worden sind. Schick' mir aber
bitte nur den Link/URL zu der zip-Datei, sonst l"auft wom"oglich
meine Mailbox "uber.


