\section{Der Aufbau der  XML-Datei}\label{sec:xml-structure} 

Der gesamte Inhalt einer Tourbeschreibung wird in einer XML-Datei abgelegt.
Ich werde in diesem Kapitel detailliert beschreiben, wie diese Datei aufgebaut
ist.

Das Wissen um den Aufbau dieser Datei ist aus zwei Gr"unden notwendig:
\begin{enumerate}
\item Solange wir keine grafische Schnittstelle zur Erfassung besitzen, mu"s
      man diese Datei per Hand erfassen. Dazu mu"s man nat"urlich wissen, wie
      sie aufgebaut ist.
\item Auch wenn wir irgenwann eine grafische Schnittstelle haben, mu"s ein(e)
      Entwickler(in), die/der das System erweitern will, wissen, wie der Aufbau
      nun exakt aussieht\footnote{Ok, es existiert eine dokumentierte DTD, aber 
      das nur am Rande\ldots}.
\end{enumerate}

Sieht man sich die XML-Datei das erste Mal an, dann schaut das alles recht 
komplex und konfus aus. Aber, ehrlich, so schwierig ist das alles gar nicht.
Fangen wir mal an\ldots

\textit{Anm.: In den XML-Beispielen sind Zeilennummern ergaenzt. Diese geh"oren 
nat"urlich \textbf{nicht} in die echte Datei!
}

\subsection{<tourguide>}

Die gesamte Tourbeschreibung wird in zwischen dem "offnenden und dem
schlie"senden \texttt{<tourguide>}-Tag eingeschlo"sen. Das Ganze sieht also
so aus:

\lstset{numbers=left,numberstyle=\tiny,stepnumber=1}
\begin{lstlisting}{}
<tourguide>
   <storage-info>
      <location />
   </storage-info>
   <general>
      ...
   <general>
   <crosspoints>
      ...
   </crosspoints>
<tourguide>
\end{lstlisting}

Im Wesentlich haben wir also drei Abschnitte:
\begin{enumerate}
\item <storage-info>
\item <general>
\item <crosspoints>
\end{enumerate}

Alle Abschnitte werden nachfolgenden genauer beleuchtet. Der
Wichtigste is \texttt{<crosspoints>}, da er die eigentlich
Beschreibung der Tour beinhaltet. Wozu sind aber die anderen
da? Fangen wir an mit\ldots

\subsection{<storage-info> (<location>)}
Wie schon erw"ahnt, erfassen wir ja unsere Tour nicht f"ur uns 
selbst: Wir kennen sie ja schon. Sondern f"ur andere.

Mal angenommen, du wohnst in Hamburg und f"ahrst "ubers Wochende
nach Buttenheim in Bayern (das MTB ist nat"urlich dabei). Am
Samstag h"attest du gute drei Stunde Zeit 'ne kleine Tour zu
fahren. Gibt's da bekannte Touren?

W"are jetzt nett, einfach ins Internet zu gehen. Das dir 
bekannte Interface zu benutzen und mal zu gucken, was da in
Buttenheim so geht.

Genau daf"ur ist das Tag \texttt{<storage-info><location></storage-info>}
vorhanden. Wie? Schauen wir uns das mal an:

\lstset{numbers=left,numberstyle=\tiny,stepnumber=1}
\begin{lstlisting}{}
<tourguide>
   <storage-info>
      <location 
          country="de" 
          region-or-state="Bavaria" 
          city="Buttenheim" 
          zip-code="96155"/>
   </storage-info>
   ...
<tourguide>
\end{lstlisting}

Alles klar? Alle Infos werden mithilfe der Attribute des Tags
\texttt{<location>} erfa"st. Wie werden diese Attribute nun 
zur Suche benutzt? Sorry, das ist (noch?) nicht Aufgabe dieses
Projekts. Sie sind erstmal da. Es liegt nun an demjenigen, der
die Suchfunktion implementiert, wie clever die erfassten Daten
genutzt werden.

Eine kurze Anmerkung zum Attribut \texttt{country}: Es sollten
die sogenannten Top Level Domains (wie sie aus dem Internet
bekannt sind) verwendet werden. Ein einfacheres und bekannteres 
System zur Angabe von L"andern existiert praktisch nicht.

Gut. Nun haben wir (mit viel Gl"uck) drei Touren gefunden.
Aber sind wir daf"ur auch fit genug? Dauert das nicht zu 
lange? Wie kommt man an den Startpunkt? Gibt's ne Karte?
Ja, das braucht man alles. Deswegen gibt es das Tag
\texttt{<general>}\ldots

\subsection{Ein Wort zu Bildern}
Hmm, den Abschnitt \texttt{<general>} hab' ich ins n"achste
Kapitel verschoben. Ich mu"s noch schnell was zu Bildern
im Allgemeinen sagen. Dauert nicht lange, ist aber wichtig
f"ur \textbf{alle} nachfolgenden Abschnitte.

Nat"urlich unterst"utzt das System die Verwendung von Bildern.
Ohne Bilder w"are das Ganze wohl nur, wenn "uberhaupt, die
H"alfte wert. Dazu m"ussen die Bilder aber auch  gefunden
werden. Aber wie?

So: In dem Verzeichnis in dem die XML-Datei, "uber die wir
hier ja reden, zu liegen kommt, existiert \footnote{Wenn das
das Grundger"ust via \texttt{ant gen-skeleton} angelegt wurde,
dann existiert es automatisch. Ansonsten mu"s es per Hand
z.B. mit dem Befehl \texttt{mkdir images} unter linux-artigen
Systemen bzw. \texttt{md images} auf einem Windows-Rechner
angelegt werden.} ein Unterverzeichnis namens \texttt{images}.
Hier hinein werden alle Bilder kopiert, die wir sp"ater
ben"otigen.

Gut, jetzt h"atten wird alle Bilder in unserem Verzeichnis.
Aber wie geben wir sie in der XML-Datei an? Auch das ist
ziemlich einfach. Schauen wir uns mal ein Beispiel an:
\lstset{numbers=left,numberstyle=\tiny,stepnumber=1}

\lstset{numbers=left,numberstyle=\tiny,stepnumber=1}
\begin{lstlisting}{}
<image name="big_drop.jpg"/>
\end{lstlisting}

Das war's auch schon. Die Bilder werden \textbf{ohne} Angabe
des Verzeichnisses (\texttt{images}) durch die Notation des 
Bildnamens eingetragen. An welchen Stellen das n"otig/m"oeglich
ist, werden wir im weiteren Verlauf erfahren.

\subsection{<general>}
Ok. Nach dem Exkurs zur Angabe von Bildern, widmen wir uns
jetzt dem n"achsten Abschnitt der XML-Datei: \texttt{<general>}.
Erstmal wieder ein Beispiel:

\lstset{numbers=left,numberstyle=\tiny,stepnumber=1}
\begin{lstlisting}{}
<tourguide>
   ...
<general 
   fitness-level="easy" 
   tech-level="medium" 
   type="round" 
   duration="01:00hrs">
   
   <name>Nuernberger Tiergarten 
         --- Kleiner Trial inklusive 'Roller Coaster'
   </name>
   <distance unit="km">10</distance>
   <desc>
       Wir starten am Tiergarten, fahren etwas uphill durch 
       die Stromschneisse, dann den 'Mini-DH' runter...
   </desc>
   <reach>
      <option name="Mit dem Auto">Auf der A3 aus Richtung 
         Wuerzburg oder Passau kommend bei Nuernberg/Moegeldorf 
	 abfahren...
      </option>
      <option name="Per Zug">Am Nuernberg Hbf aussteigen. Dann 
         mit der Strassenbahn direkt zum Tiergarten.
      </option>
   </reach>
   <maps>
      <map>Kompass, Karte 170</map>
   </maps>
   <roadmap image="map.jpg"/>
   <profile />
</general>   
   ...
<tourguide>
\end{lstlisting}

\subsubsection{Die Attribute von <general>}
Wie man aus dem Beispiel ersieht, hat <general> vier
Attribute. Diese dienen dem/r Anwender(in) dazu, zu 
entscheiden, ob die Tour f"uer ihn/sie geeignet ist.

Die Angaben hier sind nicht ganz einfach, aber wichtig.
Wenn du topfit bist, die Tour aber auch f"ur dich
hart war, dann bringt das hier zum Ausdruck. Wenn
der Downhill f"ur dich nicht ganz einfach, aber du
technisch nicht so perfekt bist, dann setze den 
\texttt{tech-level} nicht gleich auf \texttt{extrem}.

Nachfolgend die Attribute und ihre Bedeutung (was man
hierf"ur angeben kann, wird im Anschlu"s erl"autert):
\begin{enumerate}
\item fitness-level: Wie fit muss man f"uer die Tour sein?
\item tech-level: Wie hoch ist der technische Anspruch?
      Knifflige Wurzelpassagen? Extreme Downhills (mit
      fetten Drops)?
\item type: Rundtour, Single Trial, \ldots
\item duration: Dauer der Tour. Als du sie gefahren
      bist.
\end{enumerate}


So und nun zu den restlichen Tags die innerhalb von
<general> verwendet werden k"onnen oder m"ussen\ldots

\subsubsection{<distance>}
\subsubsection{<desc>}
\subsubsection{<reach>}
\subsubsection{<maps>}
\subsubsection{<roadmaps>}

\subsubsection{<profile/>}


\subsection{<crosspoints>}
Mann-o-mann, so 'ne Menge Zeugs zu tippen. Wann kommt jetzt
eigentlich die Beschreibung der Tour? Voila\ldots jetzt!

Innerhalb des <crosspoints>-Abschnitts werden alle relevanten
Wegpunkte der Tour beschrieben (und, optional, noch vieles
mehr).

Hier mal ein Beispiel:
\begin{lstlisting}{}
<tourguide>
...
<crosspoints>
   <crosspoint 
      distance="300" elevation="600" 
      direction="northwest" 
      latitude="000" longitude="000">
     <desc> Ein nordwest Punkt.</desc>
     <profile-desc>
        Die Stromschneisse... los geht's
     </profile-desc>
   </crosspoint>
   
   <track-info pavement="road">
      <anecdote distance="gesamter Weg" image="drop.jpg">
          Wenn's gar nicht mehr geht, dann einfach ab in 
	  den Wald. Der ist nicht dicht... man kann da 
	  auch gut schieben (und wird nicht gesehen ;-).
      </anecdote>
   </track-info>
   
   <crosspoint distance="200" elevation="310" 
               direction="none" 
	       latitude="000" longitude="000">
     <desc>Back At Home!</desc>
   </crosspoint>
<crosspoints>
<tourguide>
\end{lstlisting}

WTF\footnote{Engl.: \textit{What the fuck}, Ausruf des Erstaunten\ldots ;-)}?
Jaja, schaut kompliziert aus, is' aber auch blo"s mit Wasser
gekocht.

Gehen wir das mal analytisch an. Innerhalb von <crosspoints> 
scheint es ein paar andere relevante Tags zu geben. Na, also:
Richtig erkannt. Und es handelt sich dabei nur um zwei
\begin{itemize}
\item <crosspoint> und
\item <track-info>
\end{itemize}





