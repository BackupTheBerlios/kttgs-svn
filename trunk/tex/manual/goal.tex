\section{Was ist/wird/kann das?}

Ich fahre gerne mit meinem MTB durch die Gegend. Anfangs einfach 
so ins Blaue hinein, dann nahm ich schon mal 'ne Karte mit, als
ich so fit war auch mal in Gegenden vorzudringen, wo ich mich
nicht mehr so gut auskannte.

Irgendwann entdeckte ich dann, dass man im Buchhandel auch B�cher
kaufen konnte, die komplette Touren beschrieben. ``Super Sache'',
dachte ich mir.

Anfangs\ldots diese B�cher sind ja nicht ganz billig. Naja, mal
im Internet gucken: Vielleicht gibt es da ja auch was. Nicht
schlecht getippt, da gab es wirklich schon 'ne Menge Sachen. Aber
irgendwie war jede(r) Autor(in) und jede Website der Meinung, ein
eigenes System schaffen zu m�ssen.

Alles war bissl anders formatiert, bei manchen Beschreibung fehlten mir 
einfach ein paar Infos zur Tour. Alles in Allem nicht gerade das, was
einen Software-Entwickler vom Hocker rei�t: Ich brauch' Struktur und
die M�glickeit, die vorhandenen Informatioen so aufbereiten zu k�nnen,
wie ich sie gerne lese (oder zumindest verschiedene w�hlbare Einstellungen).

Jetzt wird so manch' eine(r) sagen: ``Ganz sch�n frech, da bekommt 
er die Infos f�r lau und beschwert sich auch noch dar�ber''. Stimmt!

Aber frech w�re es nur dann, wenn man sich nicht daran macht, das zu
verbessern, was man kritisiert. Das mache ich mit diesem Projekt.

\subsection{Entt�uschte Hoffnungen}
Ja, es tut mir leid, aber dies ist erst Version 0.5 des Projekts.
Ich entt�usche euch lieber gleich, bevor ich zu viele Hoffnungen
wecke.

Dieses System ist derzeit keine Software mit der man eine Tourbeschreibung
mit Hilfe einer grafischen Benutzerschnittstelle erfassen kann. Es bietet
auch keine Datenbankunterst�tzung.

Wenn ihr nicht aus dem Unix/Linux-Bereich kommt, dann fragt ihr
euch sicher was das (in das ich so viel Freizeit investiere) �berhaupt 
soll. Immerhin kann ich sagen, dass dieses System auch unter Windows
und vermutlich auch am Mac l�uft. Aber wenn ihr es nicht m�gt auf der
Kommandozeile Befehle zu tippen, dann solltet ihr mal wieder zu einem
sp�teren Zeitpunkt hierher zur�ckkommen.

Aber vielleicht lest ihr euch doch noch den n�chsten Abschnitt durch \ldots

\subsection{Ist-Zustand der Version 0.5}
Ok, nach dem letzten Abschnitt mag man sich fragen: ``Ja, was kann
es denn �berhaupt''? Gute Frage. Ich werde versuchen sie hier zu beantworten.

Ich habe erstmal versucht, eine XML-Struktur zu schaffen, die es dem
Autor der Tour erm�glicht, m�glichst alle Daten, die relevant sind,
zu erfassen. Das sind derzeit Punkte wie:

\begin{itemize}
\item Es werden Daten erfasst, die dazu dienen, die Tour in einem Online-System
      verf�gbar zu machen.
\item Erfasst wird: Technisches K�nnen, Fitness, Typ der Tour, Dauer, Gesamtlaenge.
\item Daten zur Anreise
\item Einbindung einer Karte als GIF, JPG oder PNG. Ausserdem Hinweise zu gedruckten
      Karten.
\item Detaillierte Informationen zum Streckenabschnitt.
   \begin{itemize}
       \item H�he
       \item Entfernung zum letzten Wegpunkt
       \item Breitengrad
       \item L�ngengrad
       \item Sta�enbelag/Wegbeschaffenheit (grafische Darstellung)
       \item Interessante Punkte
       \item Restaurants/Lokale/Kneipen
       \item Anekdoten
       \item Orientierung: Windrose zeigt den weiteren Weg 
       \item Orientierung: Beliebige Anzahl an Bildern beschreiben die Tour
   \end{itemize}
\item Und nat�rlich: Beschreibungen, Beschreibungen, \ldots
\end{itemize}

Ferner wird automatisch ein Profil (mit grafischer Darstellung des 
Streckenbelags) der Tour generiert. Die Ausgabe ist prinzipiell in jeder
Sprache (derzeit wird aber nur Deutsch und Englisch unterst�tzt) m�glich.
Ausserdem sollte das alles auf Windows, Linux (mein Entwicklunssystem) und
Mac (nicht getestet) laufen.

\subsection{Pl�ne}
Da gibt es \textbf{so} viele. Jetzt schreib' ich erstmal dieses Handbuch
zu Ende. Dann gibt es da so simple Dinger wie:
\begin{itemize}
\item Gedruckte Formulare zur Erfassung der Tour.
\item Tips und Tricks: Wie male ich 'ne Karte?
\item Stylesheet f�r Druckformat.
\item Mehrtagestouren.
\item \ldots
\end{itemize}

Und dann so H�mmer wie:
\begin{itemize}
\item Grafische Benutzerschnittstelle (da l�uft schon was ;-).
\item GPS-Systeme: Wie kann man die anbinden?
\item Wie kann man H�henprofile einbinden, die von anderen Systemen aufgezeichnet werden.
\item \textit{totally weird}: Audio-Ausgabe der Tour via MP3-Player.
\item \textit{bit weird}: Tips und Tricks: Wie erfasse ich die Tour mit 'nem MP3-Player?
\end{itemize}

Ihr seht schon, da kommt noch einiges. Und ich hocke hier alleine rum.
Also, wenn ihr Lust habt hier mitzuwerkln $\Longrightarrow$ bitte melden!
Ich kann jeden brauchen: Coder, Designer, Tester, Writer, Biker, Vision�r,
Hardware, \ldots

 